\documentclass{article}
\usepackage[utf8]{inputenc}
\usepackage[spanish]{babel}
\usepackage{listings}
\usepackage{graphicx}
\graphicspath{ {images/} }
\usepackage{cite}

\begin{document}

\begin{titlepage}
    \begin{center}
        \vspace*{1cm}
            
        \Huge
        \textbf{Proyecto de investigación }
            
        \vspace{0.5cm}
        \LARGE
        Nociones de la memoria del computador
            
        \vspace{2.5cm}
            
        \textbf{Mauricio Duque Quintero }
            
        \vfill
        \begin{figure}[h]
        \includegraphics[width=4cm]{Escudo-UdeA.png}
        \centering
        \label{fig:descarga}
        \end{figure}
     
        \vspace{0.8cm}
            
        \Large
        Despartamento de Ingeniería Electrónica y Telecomunicaciones\\
        Universidad de Antioquia\\
        Medellín\\
        Septiembre de 2020
            
    \end{center}
\end{titlepage}

\tableofcontents

\section{Introducción}

Por medio de este proyecto se pretende contextualizar a cerca de la memoria del computador para tener nociones precisas de que sucede internamente, al momento de interactuar con el microprocesador y los controladores, como es el caso de abrir y cerrar programas y a su vez  realizar tareas de creación, guardado y eliminación de documentos, procesamiento de información, entre otros. 

\vspace{0.5cm}

Por otra parte se analizará la organización de los diferentes tipos de memoria como estructura jerárquica, donde se pueda resaltar las ventajas que se tiene una con respecto a otra, a partir de factores como la velocidad  en el acceso a la información y su capacidad de almacenamiento.
\vspace{0.5cm}

Esto permitirá ganar habilidades y aptitudes, al momento de cumplir la función de programadores, ya que se tendrá mayor comprensión del funcionamiento interno de la memoria, para tener mayor eficiencia en la creación de proyectos. \\

\vspace{0.5cm}

\section{Memoria del Computador}
La memoria del computador es un dispositivo de alta relevancia, que tiene la capacidad de almacenar temporalmente, toda la información (datos)  y  las instrucciones necesarias que serán procesadas o ejecutadas. Usualmente se utiliza el término para referirse a dispositivos de almacenamiento temporal y alta velocidad de acceso, como lo es la memoria principal del computador.\cite{ecuredwebsite}

\section{Tipos de memoria } \label{contenido}
MEMORIA RAM ( Random Access Memory) : es la memoria más importante del computador, está dividido en celdas de memoria donde se almacenan los bits , a los cuales  se acceder directamente sin importar su posición o dirección.
	
\vspace{0.5cm}
MEMORIA ROM (Read Only Memory): Esta memoria es de solo lectura, es decir, no se puede escribir en ella. Esta memoria es imprescindible para el funcionamiento del ordenador y contiene instrucciones y datos técnicos de los distintos componentes del ordenador.
    
\vspace{0.5cm}

MEMORIA CACHE: cumple la función de almacenar direcciones concretas o extensiones del programa o programas en ejecución; además se utiliza para trabajar con los datos e instrucciones que el microprocesador ve que se utilizan con mayor regularidad.

\vspace{0.5cm}

MEMORIA VIRTUAL: es una porción de disco duro que mantiene temporalmente los fragmentos de los programas que no se está utilizando y datos en ejecución que se utilizan menos o que ocupan espacio innecesario en algún momento determinado.

\vspace{0.5cm}

DISCO DURO: Es un dispositivo de almacenamiento de datos no volátil, que contiene todos los programas del sistema y su respectiva información. Es uno de los componentes del hardware más importantes dentro de una computadora y que tiene mayor capacidad de almacenamiento.

\vspace{0.5cm}

Para realizar las definiciones anteriores se tuvo en cuenta las siguientes referencias: \cite{ecuredwebsite},\cite{tecdwebsite} y \cite{gestionwebsite}

\section{Gestión de la memoria en un computador} \label{contenido}

Al realizar cualquier tipo de acción en el computador, la memoria cumple un papel fundamental en todo momento que se procesan y almacenan los datos; la cual está en constante interacción con el microprocesador y los controladores a través del bus que es por donde viaja la información.

\vspace{0.5cm}

Al momento de enviar una orden (que se traduce a una instrucción) a través de un periférico de entrada, esta viaje en forma de pulsos eléctricos por el bus de instrucciones y  llega hasta la memoria, colocándose temporalmente en un espacio de la misma. Por consiguiente el microprocesador recibe el aviso de que existe una nueva instrucción en memoria para ser leída; esta instrucción pueden ser: abrir o cerrar un programa, cargar o guardar un archivo, etc, estos los archivos y programas generalmente se encuentran en el disco duro. Al momento de ser ejecutada la instrucción se elimina tanto del procesador como de la memoria, para que dicho espacio que ya no se utiliza, esté disponible para otra información. De esta manera el microprocesador le envía la orden a un controlador especial ubicado en la placa madre. Dicho controlador toma la aplicación o el archivo, que está almacenada en el disco duro y la lleva hasta la memoria, de forma fragmentada colocándola en un espacio vacío de la misma trayéndolos y llevándolos una y otra vez del procesador a la memoria y viceversa. Debido a que el microprocesador no posee espacio suficiente de almacenamiento para el manejo de la información. Una vez que se encuentra cargado en memoria y funcionando, el microprocesador utiliza los  recursos para poder realizar el  procesamiento indicado.

\vspace{0.5cm}

Vale resaltar que en este proceso descrito anteriormente se repite para miles de instrucciones, procesamientos de datos en el microprocesador así como millones de transferencias de porciones de datos entre la memoria y el microprocesador, las cuales se irán traduciendo a texto modificado acorde a las necesidades del usuario.\cite{tallerwebsite}

\vspace{0.5cm}


\section{Jerarquía según velocidad de la memoria} \label{contenido}
La jerarquía de una memoria a partir de su velocidad, se caracteriza por el tiempo y el proceso que realiza la memoria para acceder a la información, permitiendo mayor rendimiento en la computadora; es por esto que  el disco duro, se considera la memoria más lenta, debido a la mecánica de su funcionamiento, puesto que para acceder a la información depende de la velocidad de rotación de los discos que tiene adentro, donde un motor mecánico, en conjunto con un electroimán, mueven un cabezal rotativo sobre el que descansan los platos magnéticos en los que se graban los datos que se van a almacenar (sistema operativo, documentos, música, etc.).\cite{comwebsite}

\vspace{0.5cm}

Mientras que un módulo de memoria es más veloz, ya que permite acceder a la información almacenada temporalmente en él en lapsos de tiempo de nanosegundos, a su vez la velocidad depende del reloj del bus que conecta la memoria con el microprocesador.

\vspace{0.5cm}

Hablando específicamente de la memoria RAM que  está dividida en celdas en donde se almacenan temporalmente cada uno de los bits que componen los bytes de la información con la que trabaja el microprocesador. Se encuentra formada por un transistor y un capacitor. Mientras los capacitores sostienen los bits de información, los transistores actúan como interruptores que permiten a su controlador de memoria leer o modificar la información (los bits) que contienen cada una de las celdas de bits distribuidas en una grilla bidimensional en una matriz de columnas y filas, la intersección de una columna y una fila determinan la dirección de una celda de memoria.\cite{tallerwebsite}

\vspace{0.5cm}

No obstante existe una memoria  de mayor velocidad que la RAM y es la memoria Cache, el objetivo principal de la caché es aumentar el desempeño de recuperación de datos para evitar tener que acceder a la capa subyacente de almacenamiento, que es más lenta. Al intercambiar capacidad por velocidad, una memoria caché normalmente almacena un subconjunto de datos de forma transitoria, a diferencia de las bases de datos cuyos elementos suelen ser completos y duraderos. Cuando una aplicación intenta leer datos, generalmente de un sistema de almacenamiento de datos como una base de datos, verifica si el registro deseado ya existe en la memoria caché. Si existe, esa aplicación leerá los datos del caché, evitando que el acceso a la base de datos sea más lento. Si el registro deseado no está en la memoria caché, la aplicación lee el registro de la fuente. Cuando recupera esos datos, también los escribe en el caché para que cuando la aplicación necesite esos mismos datos en el futuro, pueda obtenerlos rápidamente del caché.\cite{awsite}

\vspace{0.5cm}

Por último es importante resaltar que a medida que aumenta la velocidad de la memoria, disminuye su capacidad de almacenamiento, convirtiendo en el disco duro la memoria  con mayor capacidad de almacenamiento y la cache con la menor capacidad.

\section{Conclusión}
El conocimiento básico del funcionamiento de la memoria, nos permite ampliar la visión a la hora de programar, comprendiendo los procesos para tener mayor optimización y mejor manejo de los recursos de dicho componente.

\vspace{0.5cm}

 Indudablemente, cada tipo de memoria que compone el computador  es fundamental para su funcionamiento, porque maneja los distintos procesos que se llevan a cabo mediante el equipo, brindándole fluidez a las tareas que estemos desarrollando. Por ende si alguna de ella es insuficiente o se encuentra mal optimizada, los procesos comenzarán a tornarse más lentos o generan dificultades al computador.
 
 \vspace{0.5cm}

No obstante, este proyecto nos permite, poner en práctica un mecanismo que será utilizado a lo largo del curso, como lo es el tema de los repositorios, donde se podrá almacenar y hacer un control de versiones de todos los proyectos, tanto de manera local, en GIT, como de manera global en GitHub. Que por ende será una herramienta que no solo servirá en la vida académica sino también laboral. Y por último la introducción al uso de la herramienta para la elaboración de documentos de textos a través de la plataforma Overleaf.

\vspace{0.5cm}


\bibliographystyle{IEEEtran}
\bibliography{references}

\end{document}
