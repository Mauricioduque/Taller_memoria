\documentclass{article}
\usepackage[utf8]{inputenc}
\usepackage[spanish]{babel}
\usepackage{listings}
\usepackage{graphicx}
\graphicspath{ {images/} }
\usepackage{cite}

\begin{document}

\begin{titlepage}
    \begin{center}
        \vspace*{1cm}
            
        \Huge
        \textbf{Taller Memoria }
            
        \vspace{0.5cm}
        \LARGE
        Nociones de la memoria del computador
            
        \vspace{2.5cm}
            
        \textbf{Mauricio Duque Quintero }
            
        \vfill
        \begin{figure}[h]
        \includegraphics[width=4cm]{Escudo-UdeA.png}
        \centering
        \label{fig:descarga}
        \end{figure}
     
        \vspace{0.8cm}
            
        \Large
        Despartamento de Ingeniería Electrónica y Telecomunicaciones\\
        Universidad de Antioquia\\
        Medellín\\
        Septiembre de 2020
            
    \end{center}
\end{titlepage}

\tableofcontents

\section{Memoria del Computador}
	La memoria del computador es un dispositivo de alta relevancia, que tiene la capacidad de almacenar temporalmente, toda la información (datos)  y  las instrucciones necesarias que serán procesadas o ejecutadas. Usualmente se utiliza el término para referirse a dispositivos de almacenamiento temporal y alta velocidad de acceso, como lo es la memoria principal del computador.

\section{Tipos de memoria } \label{contenido}
MEMORIA RAM ( Random Access Memory) : es la memoria más importante del computador, está dividido en celdas de memoria donde se almacenan los bits , a los cuales  se acceder directamente sin importar su posición o dirección.
	
\vspace{0.5cm}
MEMORIA ROM (Read Only Memory): Esta memoria es de solo lectura, es decir, no se puede escribir en ella. Esta memoria es imprescindible para el funcionamiento del ordenador y contiene instrucciones y datos técnicos de los distintos componentes del ordenador.
    
\vspace{0.5cm}

MEMORIA CACHE: cumple la función de almacenar direcciones concretas o extensiones del programa o programas en ejecución; además se utiliza para trabajar con los datos e instrucciones que el microprocesador ve que se utilizan con mayor regularidad

\vspace{0.5cm}

MEMORIA VIRTUAL: es una porción de disco duro que mantiene temporalmente los fragmentos de los programas que no se está utilizando y datos en ejecución que se utilizan menos o que ocupan espacio innecesario en algún momento determinado.

\vspace{0.5cm}

DISCO DURO: Es un dispositivo de almacenamiento de datos no volátil, que contiene todos los programas del sistema y su respectiva información. Es uno de los componentes del hardware más importantes dentro de una computadora y que tiene mayor capacidad de almacenamiento.

\vspace{0.5cm}

lo indican las reglas de la RAE. \cite{dirac}

\begin{lstlisting}
#include <stdio.h>
#define N 10
/* Block
 * comment */

int main()
{
    int i;

    // Line comment.
    puts("Hello world!");
    
    for (i = 0; i < N; i++)
    {
        puts("LaTeX is also great for programmers!");
    }

    return 0;
}
\end{lstlisting}


En la sección de teoremas (\ref{contenido})

\section{Conclusión} \label{conclulsion}

\bibliographystyle{IEEEtran}
\bibliography{references}

\end{document}
